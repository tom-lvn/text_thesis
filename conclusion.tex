\chapter{Conclusion}

This study provides a comprehensive, spatially explicit assessment of LPF fisheries in the
Mediterranean using AIS data from 2015 to 2024. Linking fishing activity to known ecological
features, such as spawning grounds and migratory pathways, offers valuable insights for spatial
management and conservation. While spatial distributions have remained broadly consistent, evolving
patterns in timing, and precise location, suggest that fisheries are responding to both regulatory
and environmental pressures. Our results underscore the importance of integrating fine-scale
spatial data into fisheries management frameworks to enhance resilience, sustainability, and
bycatch mitigation in a rapidly changing ocean, highlighting the potential of AIS data as a
valuable tool for these purposes.