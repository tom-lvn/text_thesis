\chapter{Discussion}
The fishing activity targeting large pelagic fishes (LPF) across the whole Mediterranean Sea was
analysed for the first time by means of fishing activity derived from AIS data. Strong
spatio-temporal patterns in how fishing is distributed, as well as regional hotspots were
identified. Such dynamic responses in fleet distribution are related not only to ecological cues
(e.g., due to focused fishing in spawning and feedings grounds of targeted species) but also to
socio-economic aspects (e.g., quotas and seasonal closures). Most fishing occurred within 100 km
from port in both fleets, while purse seiners operated in shallower waters when compared to
longliners. This is most likely related to the difference in fishing strategy conducted by each
fleet. AIS data proved to be a useful tool to monitor industrial LPF fleets from EU countries.
Conversely, increasing the uptake of AIS technology in the Southern Mediterranean countries would
aid in providing a fuller picture of fishing activity.

\section{Main risk areas (hotspots)}
Fishing hotspots and thus, the main risk areas for LPF, aligned with the species' known aggregation
sites, particularly around the Balearic Islands, the Tyrrhenian Sea, the Adriatic Sea, and off the
coasts of Cyprus and Malta. These hotspot regions overlap with key spawning and feeding areas
identified in previous studies \citep{medina_spawning,arocha_2007}. A prime example of this is the
migration of large (>150 kg) bluefin tuna (BFT) from the Atlantic towards their spawning regions in
the Mediterranean between mid-May until mid-July, which drives intense seasonal purse seine
activity \citep{bft_mig_med}. However, some smaller individuals remain resident in the
Mediterranean the whole year-round \citep{cermeno_15_tagging,heinisch_08}. The presence of this
all-year-round stock as well as the fisheries changing target species, might explain the extended
season of longliners. The spatio-temporal distribution of the purse seiners in the Balearic
Islands, and the Tyrrhenian Sea indicates that they might be targeting known BFT spawning grounds
\citep{medina_spawning}. Interestingly, the spawning area around the Strait of Sicily was not
consistently identified as a purse seine hotspot even though this area is frequented by fleets of
up to 5 countries. This could be because of a lack in the adoption of AIS by non-EU countries,
which is one potential handicap of AIS derived fishing activity
\citep{taconet2019global,paolo24satellite}.

\medskip

The Adriatic Sea appeared to be the most consistent purse seine hotspot identified in our study and
is frequented by purse seiners from Italy and Croatia (Fig.~\ref{fig:seines_hotspots}
and~\ref{fig:seine_effort_countries}). In contrast to the rest of the Mediterranean, where purse
seine activity is restricted to a narrow temporal window coinciding with the spawning aggregations
and the authorized fishing seasons, the Adriatic fleet operates during all seasons, particularly
Croatian-flagged vessels (Fig.~\ref{fig:seines_ridge} and~\ref{fig:seine_effort_countries}). This
continued fishing activity may be largely due to ecological aspects, i.e., the consistent presence
of smaller individuals throughout the year. In addition, socio-economic aspects may also play a key
role, with regulatory permission for Croatia and Italy to catch individuals weighing less than 30
kg for their use in tuna farms (Fig.~\ref{fig:hrv}; \citealp{hrv_farms})

\medskip

Apart from this, the purse seine hotspot in the Balearic Islands showed variation between years. In
the earlier years of the study period, it was focused in the southern part of Ibiza, and
subsequently moved more towards the north-east and the Mallorca channel \figref{fig:pss_yearly}.
This could relate to the position of a frontal region which has been shown to be spatially dynamic
over time, and important for the spawning location of BFT \citep{balbin_14,reglero_12}.

\medskip

The relatively stable longline hotspots in combination with no clear temporal trends in most, could
be likely due to the use of different types of longlines targeting various species, which are all
analysed together in our study. As a result, it is difficult to link observed fishing patterns
within each hotspot to the specific dynamic habitat use of individual species, as we were able to
do for the purse seine fishery targeting BFT around the Balearic Islands.

\medskip

Monitoring the fine-scale spatio-temporal dynamics of the longline fishery is still particularly
valuable, for example in cases where species' different ontogenetic stages occupy different
habitats as is the case for swordfish. Juveniles of this species tend to remain in coastal waters,
while adults are more associated with deeper, pelagic zones \citep{damalas_14_swo} which could also
be the case for BFT, as has been shown in other regions \citep{bft_juvenile_habitat}. This spatial
segregation by life stage offers an opportunity for more targeted monitoring and management. AIS
data can for example help identify when and where longline fishing activity occurs close to shore,
potentially indicating higher risks of juvenile bycatch. Such information could be used to assess
compliance with existing regulations, guide the implementation of spatial or seasonal closures, and
evaluate whether these are effectively protecting juvenile habitats. For swordfish, juvenile catch
rates remained high prior to the 2017 implementation of a minimum size limit of 100 cm lower jaw
fork length, and since its introduction, discards have increased as many individuals caught fall
below this threshold \citep{iccat_juvenile_catches_swo}. This not only raises conservation concerns
but also forces fishers to exert greater effort to meet quotas \citep{iccat_swo_discards}. In this
context, spatially explicit monitoring tools like AIS could help inform more adaptive and targeted
management by identifying areas with higher juvenile presence and guiding spatial or seasonal
restrictions accordingly.

\medskip

Hotspots in general however, are quite consistent along our study period (2015-2024), and the main
fishing grounds stay the same (Fig.~\ref{fig:dll_yearly} and~\ref{fig:pss_yearly}). This might be
related to the large-scale stability of the spawning and feeding grounds of the LPF targeted. This
should however be closely monitored in the future considering the big impacts of climate change on
the Mediterranean Sea \citep{climate_med,climate_med_2}. Tracking fisheries spatially and comparing
data between different years might provide early signs of changes in the distribution of these
species, especially when combined with logbook catch data (e.g., \citealp{campos_ais_logbook}). In
this context, the monitoring of environmental features in relation to fishing activity, like
distance to port and depth, might also provide additional indicators of changes in the fishing
grounds or even gears used. The clear seasonality of the purse seine fleet occurs in response to
socio-ecological cues. These fleet dynamics can be attributed to the timing of BFT migration into
the Mediterranean and the implemented seasonal closures and Total Allowable Catches decided upon by
ICCAT, which are usually reached in a very short amount of time.

\medskip

The fleets investigated in the present study and especially longliners, not only have an important
impact on target species but also on other species caught as bycatch \citep{bycatch_book}, Bycatch
in Mediterranean longline fisheries is a concern for among others, seabirds, pelagic elasmobranchs,
and sea turtles \citep{spain_swo_gear,shark_bycatch,baez_turtles_bycatch}. Drifting longlines for
example are estimated to be responsible for the bycatch of about 27 000 individual sea turtles
annually in the Mediterranean alone \citep{bycatch_book}. Loggerhead turtle \textit{Caretta
	caretta} is one of the main bycatch species in some Mediterranean longline fisheries
\citep{baez_turtles_spain,bycatch_malta} and their estimated areas of high abundance overlap with
some of the longline hotspots identified in the present work \citep{dimatteo_turtles}. There is
significant interest from both conservation and fisheries to mitigate bycatch
\citep{bycatch_humans} and there have been recent advances in mitigating turtle bycatch for some
fleets through targetted management and/or changes in the gear used (\citealp{baez_turtles_spain}
and \href{https://www.iccat.int/Documents/Recs/compendiopdf-e/2022-12-e.pdf}{ICCAT Reg. 22-12}).
The results and methodology employed here, could be a starting point to identify high risk areas
overlapping with important turtle and other bycatch species' habitats, to better target the
deployment of onboard observers. This is just an example for one species group, but further studies
could look into fine-scale spatial overlap between non-target species and longlines, to better
estimate the risk for specific species groups.

\medskip

Our analysis is likely to miss a significant number of vessels due to inconsistencies between the
vessel registries used by Global Fishing Watch (GFW) and the way the gear types are assigned. Since
GFW assigns a gear class based not only on the registry, but also on vessels' movement, so there
can be discrepancies between them, especially if vessels use multiple gears over time. This is why
we decided to include BFT purse seiners identified from national vessel registries, irrespective of
the gear type assigned by GFW\@. The national notices are however published by each country in
their respective language and are not easily accessible for cross-country analyses. The
\href{https://www.iccat.int/en/VesselsRecord.asp}{ICCAT record of vessels} contains details on
currently active vessels and their quota allocation, as well as their historic fishing
authorizations for specific species. Further improvements to this resource could be to add historic
quota allocation for each vessel to the downloadable files, and also add information on historic
quotas to the inactive vessel list, which would simplify analyses like this study and provide a
more complete picture of fishing activity.

\medskip

In addition to EU regulations, ICCAT mandates the usage of VMS for all fishing vessels above 24 m
LOA (\href{https://www.iccat.int/Documents/Recs/compendiopdf-e/2018-10-e.pdf}{ICCAT Reg. 18-10})
and fishing vessels above 15 m length overall (LOA) that are authorized to fish species managed by
ICCAT in waters beyond their flag countries jurisdiction. Additionally, all vessels authorized to
fish BFT by ICCAT are required to use VMS and share this data with ICCAT
(\href{https://www.iccat.int/Documents/Recs/compendiopdf-e/2021-16-e.pdf}{ICCAT Reg. 21-16}). This
data is however, not made public due to concerns about privacy. Incorporating this information into
stock assessments and making it publicly available could however greatly aid in increasing the
transparency of these fisheries that are often linked to Illegal, Unreported, and Unregulated (IUU)
fishing \citep{iuu_med,iccat_scrs_2008}.

\section{Data caveats}
Although AIS was not originally designed for the estimation of fishing effort and there are some
persistent challenges in deriving fishing effort from these tracking devices, they provide a very
valuable, publicly and readily available source of information for scientists, conservationists and
fishery authorities \citep{taconet2019global}. Analyses of fine-scale spatial fisheries data
derived from vessel tracking technologies like AIS or VMS have indeed been a revolution in
fisheries science and a great step towards increasing the transparency of all human activity at sea
\citep{russo_revolution,russo_revolution2}. However, to advance towards a better application of AIS
data for fisheries monitoring, various improvement could be made, such as reducing the variability
in AIS reception, which depends on a combination of terrestrial, satellite, and dynamic onboard
receivers. Additionally, temporal coverage is not uniform, and has changed over time, which can
lead to apparent changes in fishing activity that may reflect improved signal reception rather than
actual shifts in effort. GFW is however working on a dataset that quantifies AIS coverage for each
vessels' trip, which will reduce uncertainty in analysing time series data (D. Kroodsma, personal
communication, May 27, 2025).

\medskip

AIS data is also biased towards vessels larger than 15 m and countries where its use is mandatory,
limiting its ability to fully represent fishing activity in regions dominated by small-scale fleets
or in fleets operating without AIS\@. This bias was evident for example, in the lack of French
longline vessels from the Gulf of Lions which are mainly below 15 m LOA \citep{french_longlines}
and in the fact that there seems to be no purse seine or longline activity in the southern
Mediterranean. Additionally, differences in gear types and fishing strategies are not always
captured by AIS-based models. For example, longliners and purse seiners measure effort differently
(e.g., number of hooks vs.\ number of sets), while GFW's general fishing detection model uses
fishing hours as a unified metric. This can lead to potential overestimation of absolute effort,
though the geographic accuracy of fishing locations remains high \citep{bias}.

\medskip

Finally, AIS alone cannot distinguish between métiers, which limits species-specific analyses.
Future research could combine AIS with catch or logbook data to estimate catch per unit effort
(CPUE), allowing a more accurate evaluation of fishing operations and thus, resource abundance
\citep{niu_ais_cpue}.

\medskip

Despite the aforementioned limitations, and considering the precise number of fishing hours with
caution, this study offers a reliable picture of the broad distribution of fishing in almost the
whole Mediterranean Sea. These findings can contribute to the sustainable exploitation of migratory
LPF\@. The methods used here could be combined with other data like log books to get a more
detailed information on changes in relative abundance of target species. This would provide more
details on fishing fleets to management, which is necessary for the implementation of
ecosystem-based management strategies. The areas we identified as fishing hotspots, and potentially
also the migratory routes between them, are essential areas for migratory species and could be
prime targets for management purposes, for example to implement seasonal or spatial closures.
\cite{relano_pauly} proposed to protect migratory LPF through \textit{Blue Corridors}, essentially
protected areas along the species' migratory pathways, and our findings could help in identifying
these areas. Mapping fishing intensity also has other direct implications for conservation, such as
protecting bycatch species, of which rates can be substantial for Mediterranean longliners.