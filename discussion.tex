\chapter{Discussion}
Our analysis of fishing activity targetting LPF in the Mediterranean revealed that a total of 213
unique longline and purse seine vessels that accounted for an average of over 120 000 fishing hours
annually, with concentrated effort along the Spanish Mediterranean coast, Sardinia, Sicily, Malta,
the Adriatic, and around Cyprus. These areas correspond to some of the main known spawning and
feeding grounds of LPF in the Mediterranean. Supporting this, both fleets also exhibited strong
seasonal patterns, with longline fishing peaking in summer and purse seine fishing concentrated in
spring driven by the migration towards spawning grounds. Most fishing activity occurred within 100
km of ports, with purse seiners operating in shallower waters compared to the broader depth range
of longliners. Monitoring the depth or distance to port of individual fleets could provide insights
into potential shifts in distribution related to climate change for example. Notably, the majority
of vessels detected by AIS were flagged to EU countries, especially Italy, and AIS data suggested
an under-representation of non-EU fishing activity when compared to official catch and vessel
records from ICCAT and GFCM.

\section{Main risk areas (hotspots)}
Considering that fisheries of highly migratory LPF occur around aggregations of these animals they
should generally follow patterns in the distribution of the target species which aggregate mainly
for spawning but also in feeding areas. The fishing hotspots identified generally correspond to
spawning and feeding grounds of LPF\@.

\medskip

In the Mediterranean, most research on these important areas has focused on BFT\@. Large BFT > 25
kg come to the Mediterranean Sea for spawning and stay from mid-May until mid-July while some
individuals have been shown to reside in the Mediterranean the whole year \citep{bft_mig_med}. They
are caught by both longlines and purse seines \citep{iccat_bft_summary}. Purse seine hotspots
around the Balearic Islands and in the Tyrrhenian Sea correspond to known spawning grounds of BFT
\citep{medina_spawning}. Notably however, the spawning ground in the strait of Sicily was not
consistently identified as a hotspot, probably due to a lack of purse seiners equipped with AIS
from southern Mediterranean countries. The most consistent hotspot in our study was the Adriatic
Sea where purse seine activity is not limited to spring unlike the other hotspots. One explanation
for this could be that a resident BFT population frequents this area throughout the year and
catches demonstrate mainly individuals below reproductive age in this area
\citep{iccat_bft_catches_adria}. This population stays in the Adriatic Sea and only migrates out of
the Mediterranean Sea after reaching maturity around the age of 8 years. Aggregations of BFT in the
Adriatic appear to not be related to spawning but to feeding and Croatia is currently the only
country that is allowed to catch BFT below > 30 kg and use them in fattening farms. This explains
the year-round fishing activity by the Croatian purse seine fleet which is also reflected in the
catches reported to ICCAT \figref{fig:hrv}.

\medskip

Swordfish migratory behaviour is highly complex but local increases in density appear to be
dictated by seasonal migration to spawning areas when the 24°C isotherm occurs
\citep{arocha_2007,palko1981swordfish}. In the Mediterranean, swordfish are known to spawn from
June to August between the Balearic Islands and east of the Strait of Gibraltar, in the Tyrrhenian
Sea and the Strait of Messina, the Levantine Sea (between Crete and Cyprus), and the Gulf of
Taranto in the Ionian Sea (see Fig. 5 in \citealp{arocha_2007}). Juveniles and adults occupy
different habitats with juveniles being more associated to coastal areas and adults more to pelagic
waters \citep{damalas_14_swo}. Swordfish are one of the main targets of the Mediterranean longline
fleets. The largest persistent fishing hotspot that was identified spans from west of the Balearic
Islands until the east of the Strait of Gibraltar, overlapping with the spawning ground identified
there. Other persistent hotspots overlapping with known spawning areas are located in the
Tyrrhenian Sea close to Sicily and to the east of Sicily in the Ionian Sea. Catches of juveniles
are generally quite high in the Mediterranean and one of the management regulations currently in
place for swordfish is a minimum size limit (100 cm [ICCAT Rec 16-05],
\citealp{iccat_juvenile_swo_ortiz}). This minimum size limit has however led to a higher discard
rate and higher effort, as since then, more individuals are discarded as they fall under the
minimum size, which in turn increases the effort necessary to reach the quota
\citep{iccat_swo_discards}. The swordfish stock in the Mediterranean is currently still considered
overfished but a seasonal closure has been implemented from January to the end of March that also
extends to the albacore fleet during October and November [ICCAT Rec 16-05].

\medskip

Capture as non-target species in Mediterranean longline fisheries is a concern for seabirds,
pelagic elasmobranchs, and sea turtles. Drifting longlines are estimated to be responsible for the
bycatch of about 27 000 individual sea turtles annually in the Mediterranean alone
\citep{bycatch_book}. The same authors also identified the most important parameters when
investigating capture rates of loggerhead turtles to be fishing effort, including detailed
information on the métier and the time spent fishing. The longline hotspots that were identified in
our analysis, show overlap with estimates of high relative abundance of loggerhead turtles
\textit{Caretta caretta} most notably in the Tyrrhenian Sea, the Baleric Islands, and south of
Malta \citep{dimatteo_turtles,bycatch_malta}. \cite{bycatch_malta} analysed the longline fishery
around Malta and reported that \textit{Caretta caretta} makes up 40\% of the total catch in terms
of individual animals. Considering that this area sees some of the highest longline activity
identified in this study and shows an increasing trend, closer monitoring on bycatch rates is
necessary. Although there appear to be recent advances in reducing bycatch of turtles for example
in Spain, these are driven by changes in métier (mainly due to deeper sets that sink faster) and
not through targetted management. Future economically driven changes in métier could thus, easily
reverse this decrease. The results and methodology employed here, could be a starting point to
identify high risk areas overlapping with important turtle habitat, to better target the deployment
of onboard observers. Further studies could look into fine-scale spatial overlap between non-target
species and longlines, to better estimate the risk for specific species groups.

\section{Trends}
The spawning of BFT in the Mediterranean has been shown to be highly reliant on the position of a
frontal region where newly arrived, less saline Atlantic water meets more saline, resident
Mediterranean waters around the Balearic Islands \citep{reglero_12}. The position of this front is
however, variable between years \citep{balbin_14}. The position of the purse seine hotspots in
which the individuals migrating from the Atlantic to spawn are targetted, should thus follow the
position of this front between years. In the present analysis we were able to show that the
location of fishing in this hotspot changes between years and following the variable front is a
possible explanation for this.

The clear seasonality of the purse seine fleet can be attributed to the timing of BFT migration
into the Mediterranean and the implemented TAC's which are reached in a very short amount of time.
The longline season appears to be expanding over time which could be related to temperatures being
higher for longer due to climate change or an increase in fishing activity is necessary to reach
the TAC's. In general however, there are no large-scale shifts in fishing grounds and they stay
mostly in similar areas from 2015 until 2024. This finding shows that large variation in spawning
grounds is rare for the LPF targetted by the gears analysed here.

Some longline fleets analysed here have had changes in the gear that is used recently. Italian and
Spanish longline fleets targetting swordfish are now using mostly mid-water longlines instead of
traditional surface longlines \citep{swo_gear_italy,spain_swo_gear}. As this gear occurs in deeper
waters, a potential way of monitoring changes would be the depth in which a fleet operates.

\section{Data caveats}
AIS technology was not designed for the estimation of fishing effort and thus, there are some
persistent challenges in deriving fishing effort from these tracking devices. First, there are
differences in the coverage of different AIS receivers. Global Fishing Watch receives raw AIS
signals from terrestrial receivers, satellite receivers, and \textit{Dynamic AIS}, which are
receivers onboard thousands of large vessels that travel along major shipping routes and relay AIS
signals they receive to satellites. Reception of these signals is however, not homogenous in all
places, especially in areas with high vessel density like the Mediterranean. Additionally, there
have been changes in the AIS sources that underlie the calculation of fishing hours from GFW
throughout our study period. This could potentially mask real variation in fishing hours and lead
to an apparent increase in activity while in reality only coverage increased. This is the reason
why time series derived from this data is difficult to analyse, however GFW is currently working on
an additional dataset that quantifies AIS coverage for each vessels' trip which would enable a more
detailed, quantitative analysis of differences in fishing activity over time. Second, AIS is biased
towards larger vessels (> 15 m) and towards countries in which this technology is mandatory as
shown in this study. This is especially relevant in the Mediterranean, as the majority of fishing
vessels belong to the small-scale fleet and an important part of Total Allowable Catches are held
by non-EU countries. French longliners for example operate in the Gulf of Lions but we were not
able to detect any French longline vessel through AIS, likely because they are < 15 m
\citep{french_longlines}. Third, The fishing detection model used by GFW is a general model that
analyses different métiers and even different gear types under the same \textit{effort} metric
which is fishing hours. These hours include multiple fishing-related activities. This can be for
example, setting gear, soak time, gear hauling but also searching for fish. The two main gear types
in our study usually use different measures of effort. For pelagic longlines for example the number
of hooks or hauls and for tuna purse seines for instance the number of sets. This can result in
overestimation of fishing hours as recently demonstrated by \cite{bias}. The authors showed that
GFW data can overestimate the real time spent fishing by between 30\% and 380\% but also that the
geographic position of hauls was identical based on self-sampling data from two pelagic trawling
fleets. This illustrates that the results of our studies should not be interpreted as an absolute
measure of fishing effort exerted but rather as a relative quantitative measure that is indicative
of broad-scale trends especially considering the large time frame for which data was analysed.
Fourth, AIS is not able to detect different métiers as small changes in gear type for example can
mean that a different species is being targetted. This is one of the limitations here, as all three
LPF are targetted by different types of longlines, but we are not able to differentiate between the
fishing hours of each métier based on AIS\@. Further research could improve this for example
through combining logbook data on catches with AIS data to estimate catch per unit effort (CPUE),
allowing a more accurate evaluation of fishing operations and thus, resource abundance
\citep{niu_ais_cpue}.

\medskip

The results of this study demonstrate the broad distribution of fishing in almost the whole
Mediterranean Sea and can contribute to the sustainable exploitation of migratory LPF\@. The
methods used here could be combined with other data like log books to get a more detailed
information on changes in relative abundance of target species. This information would provide more
detailed information to management which is necessary for the implementation of ecosystem-based
management. The areas we identified as fishing hotspots and potentially also the migratory paths
between them could be prime targets for management for example to implement seasonal or spatial
closures. \cite{relano_pauly} proposed to protect migratory LPF through \textit{Blue Corridors},
essentially protected areas along the species' migratory pathways. Our findings could help in
identifying these areas. Mapping of the intensity of fishing does also have direct implications for
conservation, as bycatch rates can be quite high for pelagic longliners in the Mediterranean.