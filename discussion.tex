\chapter{Discussion}
Our analysis of fishing activity targetting LPF in the Mediterranean revealed that a total of 213
unique longline and purse seine vessels that accounted for an average of over 120 000 fishing hours
annually, with concentrated effort along the Spanish Mediterranean coast, Sardinia, Sicily, Malta,
the Adriatic, and around Cyprus. These areas correspond to some of the main known spawning and
feeding grounds of LPF in the Mediterranean. Supporting this, both fleets also exhibited strong
seasonal patterns, with longline fishing peaking in summer and purse seine fishing concentrated in
spring driven by the migration towards spawning grounds. Most fishing activity occurred within 100
km of ports, with purse seiners operating in shallower waters compared to the broader depth range
of longliners. Monitoring the depth or distance to port of individual fleets could provide insights
into potential shifts in distribution related to climate change for example. Notably, the majority
of vessels detected by AIS were flagged to EU countries, especially Italy, and AIS data suggested
an under-representation of non-EU fishing activity when compared to official catch and vessel
records from ICCAT and GFCM.

\section{Main risk areas (hotspots)}
Fishing hotspots and thus, the main risk areas for LPF, aligned with their known aggregation sites,
particularly around the Balearic Islands, the Tyrrhenian Sea, the Adriatic Sea, and off the coasts
of Cyprus and Malta. These regions overlap with key spawning and feeding areas identified in
previous studies.

\medskip

For large BFT (> 150 kg), spawning migrations from the Atlantic into the Mediterranean drive
intense seasonal purse seine activity when they enter the Mediterranean around mid-May and remain
until mid-July to spawn \citep{bft_mig_med}. There is however, also individuals that stay resident
in the Mediterranean the whole year-round. This has been documented for both the eastern and
western Mediterranean, and these resident individuals are mostly smaller
\citep{cermeno_15_tagging,heinisch_08}. BFT in the Mediterranean are targetted by both drifting
longlines and purse seines. The purse seine hotspots identified around the Balearic Islands and in
the Tyrrhenian Sea coincide with known spawning grounds \citep{medina_spawning}. Interestingly, the
spawning area in the Strait of Sicily was not consistently identified as a hotspot, likely because
of a lack in AIS adoption in non-EU countries that fish there. \medskip

The Adriatic Sea appeared to be the most consistent purse seine hotspot identified in our study and
is frequented by purse seiners from Italy and Croatia (Fig.~\ref{fig:seines_hotspots}
and~\ref{fig:seine_effort_countries}). Purse seine activity occurred here all in all seasons
particularly by the Croatian fleet (Fig.~\ref{fig:seines_ridge}
and~\ref{fig:seine_effort_countries}). This might be explained by the presence of smaller
individuals in this region throughout the whole year due to the fact that Croatia and Italy are the
only countries permitted to catch individuals < 30 kg for the use in tuna farms. Although the
season is officially restricted from 15 May until 15 July\citep{iccat_mp}, Croatia has declared
purse seine catches outside this window in our study period \figref{fig:hrv}.

\medskip

The spawning of BFT in the Mediterranean has been shown to be highly reliant on the position of a
frontal region where newly arrived, less saline Atlantic water meets more saline, resident
Mediterranean waters around the Balearic Islands \citep{reglero_12}. The position of this front is
however, variable between years \citep{balbin_14}. The centroid of the purse seine hotspots in
which the individuals migrating from the Atlantic to spawn are targetted, should thus follow the
position of this front between years. In the present analysis we were able to show that the
location of fishing in this hotspot changes between years and following the variable front is a
possible explanation for this.

\medskip

Swordfish show more complex migration patterns, but spawning migration is linked to thermal fronts,
particularly the 24\textdegree C isotherm \citep{palko1981swordfish,arocha_2007}. Corresponding to
this, they spawn from June to August in areas west of the Balearic Islands until the Strait of
Gibraltar, in the Tyrrhenian Sea and the Strait of Messina, the Levantine Sea, and the Gulf of
Taranto in the Ionian Sea (see Fig. 5 in \citealp{arocha_2007}). They are targetted mainly by the
longline fleet and corresponding hotspots overlapped closely with these spawning areas,
particularly in the western Mediterranean and the Tyrrhenian region.

\medskip

One critical aspect of the swordfish fisheries in the Mediterranean is the differential habitat use
of juveniles and adults where juveniles are more associated to coastal waters, and adults more to
deeper, pelagic waters \citep{damalas_14_swo}. This distinction has important implications for
their ecology but also fisheries management. Juvenile catch rates in the Mediterranean have been
shown to be at high level before a minimum size of 100 cm lower jaw fork length was established in
2017 through ICCAT Regulation 16-05 \citep{iccat_juvenile_catches_swo,iccat_juvenile_swo_ortiz}.
This minimum size limit has however, led to an increase in discards, as many caught individuals
fall below the threshold which in turn means that fishers need to exert more effort to reach
quotas, as these undersized catches are often discarded \citep{iccat_swo_discards}. Monitoring the
fine-scale spatio-temporal extent of these fisheries could thus, prove as a valuable tool in
informing spatial or seasonal closures and assessing compliance where they are already in place.
The swordfish stock in the Mediterranean is currently still considered overfished but a seasonal
closure has been implemented from January to the end of March that also extends to the albacore
fleet during October and November [ICCAT Rec 16-05].

Monitoring changes in the depth and season at which the swordfish fishery operates, could also
indicate changes in the type of longline used, as different longlines are used at different depths
and at different times of the season (see \cite{spain_swo_gear} for a full overview of longlines
used by the Spanish fleet).

Probably due to the usage of different longlines that target different species that are all
analysed together for our analysis, our study does not show clear trends in most hotspots and is
limited in linking changes inside each longline hotspot to species specific characteristic
(dynamic) habitats, like we were able to do for the purse seine hotspot around the Balearic
Islands.

\medskip

The clear seasonality of the purse seine fleet can be attributed to the timing of BFT migration
into the Mediterranean and the implemented TAC's which are reached in a very short amount of time.
The longline season appears to be expanding over time which could be related to temperatures being
higher for longer due to climate change or an increase in fishing activity is necessary to reach
the TAC's. In general however, there are no large-scale shifts in fishing grounds and they stay
mostly in similar areas from 2015 until 2024. This finding shows that large variation in spawning
grounds is rare for the LPF targetted by the gears analysed here. In case of the longline fleet, it
is however not possible to differentiate between types of longlines and target species which does
not allow us to make fine-scale assessments of changes in this fleet.

\medskip

Capture as non-target species in Mediterranean longline fisheries is a concern for seabirds, marine
mammals, pelagic elasmobranchs, and sea turtles
\citep{spain_swo_gear,mammal_bycatch,shark_bycatch,baez_turtles_bycatch,bycatch_book}. Drifting
longlines are estimated to be responsible for the bycatch of for example about 27 000 individual
sea turtles annually in the Mediterranean alone \citep{bycatch_book}. The same authors also
identified the most important parameters when investigating capture rates of loggerhead turtles to
be fishing effort, including detailed information on the métier and the time spent fishing. The
longline hotspots that were identified in our analysis, show overlap with estimates of high
relative abundance of loggerhead turtles (\textit{Caretta caretta}) most notably in the Tyrrhenian
Sea, the Baleric Islands, and south of Malta \citep{dimatteo_turtles,bycatch_malta}.
\cite{bycatch_malta} analysed the longline fishery around Malta and reported that \textit{Caretta
	caretta} makes up 40\% of the total catch in terms of individual animals. Considering that this
area sees some of the highest longline activity identified in this study and shows an increasing
trend, closer monitoring on bycatch rates is necessary. Although there appear to be recent advances
in reducing bycatch of turtles for example in Spain, these are driven by changes in métier (mainly
due to deeper sets that sink faster) and not through targetted management
\citep{baez_turtles_spain}. Future economically driven changes in métier could thus, easily reverse
this decrease. The results and methodology employed here, could be a starting point to identify
high risk areas overlapping with important turtle habitat, to better target the deployment of
onboard observers. This is just an example for one species group, but further studies could look
into fine-scale spatial overlap between non-target species and longlines, to better estimate the
risk for specific species groups.

\medskip

A significant number of vessels are likely to be missed due to inconsistencies and poor maintenance
of vessel registries. Improving the quality of these registries is essential to better identify and
include vessels in future analyses. Furthermore, the lack of historic vessel authorization data in
a downloadable format limits retrospective analyses at the level of individual vessels. Since 2018,
the usage of VMS is mandated by ICCAT for fishing vessels above 15 m length overall LOA that are
authorized to fish in waters beyond their flag countries jurisdiction, and for all fishing vessels
above 24 m LOA. Additionally, all bluefin tuna vessels authorized for fishing by ICCAT are required
to use VMS and share their data with ICCAT. This data is however, not made publicy due to concerns
about privacy. Incorporating this information into stock assessments and making it publicly
available could however greatly aid in increasing the transparency of these fisheries that are
often linked to Illegal, Unreported, and Unregulated (IUU) fishing \citep{iccat_bft_summary} (need
more citations).

\section{Data caveats}
AIS technology was not designed for the estimation of fishing effort and thus, there are some
persistent challenges in deriving fishing effort from these tracking devices. First, there are
differences in the coverage of different AIS receivers. Global Fishing Watch receives raw AIS
signals from terrestrial receivers, satellite receivers, and \textit{Dynamic AIS}, which are
receivers onboard thousands of large vessels that travel along major shipping routes and relay AIS
signals they receive to satellites. Reception of these signals is however, not homogenous in all
places, especially in areas with high vessel density like the Mediterranean. Additionally, there
have been changes in the AIS sources that underlie the calculation of fishing hours from GFW
throughout our study period. This could potentially mask real variation in fishing hours and lead
to an apparent increase in activity while in reality only coverage increased. This is the reason
why time series derived from this data is difficult to analyse, however GFW is currently working on
an additional dataset that quantifies AIS coverage for each vessels' trip which would enable a more
detailed, quantitative analysis of differences in fishing activity over time. Second, AIS is biased
towards larger vessels (> 15 m) and towards countries in which this technology is mandatory as
shown in this study. This is especially relevant in the Mediterranean, as the majority of fishing
vessels belong to the small-scale fleet and an important part of Total Allowable Catches are held
by non-EU countries. French longliners for example operate in the Gulf of Lions but we were not
able to detect any French longline vessel through AIS, likely because they are < 15 m
\citep{french_longlines}. Third, The fishing detection model used by GFW is a general model that
analyses different métiers and even different gear types under the same \textit{effort} metric
which is fishing hours. These hours include multiple fishing-related activities. This can be for
example, setting gear, soak time, gear hauling but also searching for fish. The two main gear types
in our study usually use different measures of effort. For pelagic longlines for example the number
of hooks or hauls and for tuna purse seines for instance the number of sets. This can result in
overestimation of fishing hours as recently demonstrated by \cite{bias}. The authors showed that
GFW data can overestimate the real time spent fishing by between 30\% and 380\% but also that the
geographic position of hauls was identical based on self-sampling data from two pelagic trawling
fleets. This illustrates that the results of our studies should not be interpreted as an absolute
measure of fishing effort exerted but rather as a relative quantitative measure that is indicative
of broad-scale trends especially considering the large time frame for which data was analysed.
Fourth, AIS is not able to detect different métiers as small changes in gear type for example can
mean that a different species is being targetted. This is one of the limitations here, as all three
LPF are targetted by different types of longlines, but we are not able to differentiate between the
fishing hours of each métier based on AIS\@. Further research could improve this for example
through combining logbook data on catches with AIS data to estimate catch per unit effort (CPUE),
allowing a more accurate evaluation of fishing operations and thus, resource abundance
\citep{niu_ais_cpue}.

\medskip

The results of this study demonstrate the broad distribution of fishing in almost the whole
Mediterranean Sea and can contribute to the sustainable exploitation of migratory LPF\@. The
methods used here could be combined with other data like log books to get a more detailed
information on changes in relative abundance of target species. This information would provide more
detailed information to management which is necessary for the implementation of ecosystem-based
management. The areas we identified as fishing hotspots and potentially also the migratory paths
between them could be prime targets for management for example to implement seasonal or spatial
closures. \cite{relano_pauly} proposed to protect migratory LPF through \textit{Blue Corridors},
essentially protected areas along the species' migratory pathways. Our findings could help in
identifying these areas. Mapping of the intensity of fishing does also have direct implications for
conservation, as bycatch rates can be quite high for pelagic longliners in the Mediterranean.