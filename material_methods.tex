\chapter{Material and Methods}

\section{Data sources}
\subsection{Fishing hours}
Data on fishing hours was obtained from Global Fishing Watch (GFW). This non-profit organization provides a global dataset of estimated fishing activity
derived from AIS data \citep{gfw_dataset}. 
They process data from over 190,000 unique AIS transmitters, each assigned a unique Maritime Mobile Service Identity (MMSI).
These AIS devices broadcast a vessel's location as frequently as every 2 seconds \citep{kontasvesselupdate, taconet2019global}. Along with the exact location,
each AIS transmission includes a timestamp, speed, and heading of the vessel. GFW then analyses these positional data points, to infer fishing activity, via two 
different Convolutional Neural Networks (CNN's), which are described in detail in \cite{Kroodsma18}. 

\bigskip

A first CNN classifies fishing vessels into one of sixteen fishing gear categories \figref{fig:vessel_classification} and 
predicts vessel characteristics such as length, tonnage, and engine power. This deep-learning model is trained on
a dataset of vessels matched to official vessel registries which are \textit{known} fishing vessels. Vessels without
gear information in the registries used are assigned a gear if their movement patterns resemble those of a known
vessel class. A second CNN classifies every obtained AIS position as either fishing or non-fishing based on characteristic fishing movement (\citeauthor{Kroodsma18}, \citeyear{Kroodsma18}; Fig.~\ref{fig:workflow}). 
These individual fishing events are then aggregated into grid cells spanning either 0.1° or 0.01° on a side. For the present analysis, 0.1° resolution was chosen,
covering the study area (Mediterranean Sea; Fig~\ref{fig:study_area}) in sufficient detail.

\bigskip

Unique vessels are identified based on their MMSI number and the dataset contains information
on the vessel's registration and flag country. The gear class estimated by the CNN is also compared with the information
from different vessel registries, such as the EU fleet register or the ICCAT record of vessels. In case the derived vessel class
does not match to the one in the registry, GFW assigns the broadest gear type that allows for agreement. If for example,
a vessel is registered as a purse seiner but is inferred to be a tuna purse seiner, it would ultimately be assigned to the
purse seiner class \figref{fig:vessel_classification}.
\medskip
\begin{figure}[htbp]
\centering
\resizebox{\textwidth}{!}{%
\begin{forest}
    [Fishing, fill=grayBase, for tree={grow=east,
        draw,
        rounded corners,
        minimum height=1.2em,
        text width=4cm,
        align=center,
        parent anchor=east,
        child anchor=west,
        edge path={\noexpand\path [draw, \forestoption{edge}]
            (!u.parent anchor) -- +(5pt,0) |- (.child anchor)\forestoption{edge label};},
        l sep+=10pt,
        s sep+=10pt}
        [Squid Jigger, fill=grayLeaf]
        [Drifting Longlines, fill=grayLeaf]
        [Pole And Line, fill=grayLeaf]
        [Trollers, fill=grayLeaf]
        [Fixed Gear, fill=blueBase
            [Pots And Traps, fill=blueLeaf]
            [Set Longlines, fill=blueLeaf]
            [Set Gillnets, fill=blueLeaf]]
        [Trawlers, fill=grayLeaf]
        [Dredge Fishing, fill=grayLeaf]
        [Seiners, fill=greenBase
            [Purse Seines, fill=greenBase
                [Tuna Purse Seines, fill=greenLeaf]
                [Other Purse Seines, fill=greenLeaf]]
            [Other Seines, fill=greenLeaf]]]
    \end{forest}
} % end resizebox
\medskip
\caption{Hierarchy of fishing gears recognised by GFW}
\label{fig:vessel_classification}
\end{figure}

\subsection{ICCAT catch data}
Data on catches of large-pelagic species in the Mediterranean is openly available from the International Commission for the 
Conservation of Atlantic Tunas (ICCAT). Nominal catch data was obtained for the period 2015-2023 and filtered for the major 
large pelagic species like tunas and billfishes \citep{iccat_catches}.


\subsection{Bluefin Tuna vessels}
Individual purse seine vessels that were assigned a bluefin tuna quota for 2024 were obtained from the corresponding national notices. In France,
quotas are allocated and published by the \cite{registry_france}. In Spain,
by the \cite{registry_spain} and in Italy by the \cite{registry_italy}. These countries were chosen exemplary because in the period from 2015 to 2023,
they made up more than half of the total landings of BFT for all contracting parties to ICCAT, of which close to 90\% are fished with purse seine nets \citep{iccat_bft_summary}.
Since the national notices did not contain the vessels MMSI number (which is the identifier used by GFW), the vessel names and registration numbers were
cross-referenced with the Europan Fleet Register to obtain it \citep{eu_fleet_register}. 


\section{Data filtering}
All data filtering was conducted using R (v4.4.1; \citeauthor{r_language}, \citeyear{r_language}) within the RStudio environment \citep{rstudio}.
The GFW dataset was first cropped and adjusted to a shapefile of the Mediterranean Sea \figref{fig:med}, obtained from the General Fisheries
Commission for the Mediterranean (GFCM). Subsequent filtering was based on the assigned gear type and GFW registry
information. Only entries with the gear type drifting longlines or tuna purse seines, and that were registered with
ICCAT were retained \figref{fig:workflow}. These gear types were chosen as they are the main \textit{metiers} involved in the exploitation of large-pelagic species in the Mediterranean.
Since 2015, on average, 95\% of BFT's, and 99\% of albacore tunas are caught using either longlines or purse seine nets and close to 100\% of swordfish catches use longlines \citep{iccat_bft_summary,iccat_alb_summary,iccat_swo_summary}
The GFW dataset is available from the year 2012 until 2024. However, only entries from 2015 were retained, in order to avoid masking \textit{real} fishing dynamics with the increase in
adoption of AIS devices, which only became mandatory in the EU in 2014 for all vessels > 15 m in length \citep{ec2011directive}. It is estimated however, that in the Mediterranean the EU fishing fleet >15 m is 100\% equipped with AIS since 2018 \citep{taconet2019global}.

\begin{figure}[htp]
      \begin{center}
        \begin{tikzpicture}[
            node distance=0.9cm,
            gfwbox/.style={rectangle, draw, rounded corners, text width=4.2cm, align=center, minimum height=1cm, fill=lightblue},
            yourbox/.style={rectangle, draw, rounded corners, text width=4.2cm, align=center, minimum height=1cm, fill=lightgray},
            gfwfinal/.style={rectangle, draw, rounded corners, text width=4.2cm, align=center, minimum height=1cm, fill=highlightblue},
            yourfinal/.style={rectangle, draw, rounded corners, text width=4.2cm, align=center, minimum height=1cm, fill=highlightgray},
            gfwheader/.style={font=\bfseries\large, text width=5.5cm, align=center},
            yourheader/.style={font=\bfseries\large, text width=4.2cm, align=center},
            arrow/.style={-{Stealth}, thick}
        ]
        
        % GFW Header (wider)
        \node[gfwheader] (gfwheader) at (0,0) {Global Fishing Watch Processing};
        %GFW logo
        \node[anchor=north west] at ([xshift=-1.5cm,yshift=2cm]gfwheader.north east) {\includegraphics[width=3cm]{Figures/logos/GFW_logo.png}};
        % GFW Steps
        \node[gfwbox, below=of gfwheader] (raw) {Raw AIS positions};
        \node[gfwbox, below=of raw] (cnn1) {CNN 1: Vessel classification};
        \node[gfwbox, below=of cnn1] (cnn2) {CNN 2: Detect fishing activity};
        \node[gfwfinal, below=of cnn2] (grid) {Gridded fishing hours per MMSI/trip};
        
        \draw[arrow] (raw) -- (cnn1);
        \draw[arrow] (cnn1) -- (cnn2);
        \draw[arrow] (cnn2) -- (grid);
        
        % Your Workflow Header
        \node[yourheader, right=3cm of gfwheader] (yourheader) {Filtering Steps};
        % R Logo in top-right corner
        \node[anchor=north west] at ([xshift=-1cm,yshift=1.3cm]yourheader.north east) {\includegraphics[width=1.5cm]{Figures/logos/R_logo.png}};

        % Your Steps
        \node[yourbox, below=of yourheader] (crop) {Crop to Mediterranean Sea area};
        \node[yourbox, below=of crop] (gear) {Gear types: \\ Drifting longlines \& tuna purse seines};
        \node[yourbox, below=of gear] (registry) {GFW registry \\ includes ICCAT};
        \node[yourbox, below=of registry] (national) {Manually adding known purse seiners \\ based on national quota allocation};
        \node[yourbox, below=of national] (port) {Remove cells <1 km \\ from port};
        \node[yourfinal, left=2cm of port] (final) {Final dataset used};
        
        \draw[arrow] (crop) -- (gear);
        \draw[arrow] (gear) -- (registry);
        \draw[arrow] (registry) -- (national);
        \draw[arrow] (national) -- (port);
        \draw[arrow] (port) -- (final.east);
        
        % Connector from GFW to your steps
        \draw[arrow] (grid.east) -- ++(0.5,0) |- (crop.west);
        
        \end{tikzpicture}
        \end{center}        
\caption{Global Fishing Watch data processing pipeline and filtering steps to obtain the final dataset used in this study. CNN = Convolutional Neural Network}
\label{fig:workflow}
\end{figure}

\medskip

In case of the tuna purse seiners, these filtering steps also removed some vessels which are known to be fishing for bluefin tuna based on 
national quota allocations (obtained for Spain, France, and Italy). These vessels were
removed by the filtering steps either because GFW assigns them a different gear type, or they are not present in the
registry information from ICCAT that GFW uses. For the present analysis, these vessels were thus, manually included after comparison with the information from the national
vessel registries.

\medskip

Irregular vessel movement patterns occurring in or near ports can falsely resemble fishing activity and should therefore be excluded when estimating fishing effort. 
These movements include cruising towards the harbour to land catches or vessel maintenance \citep{souza}. Thus,
to remove these areas from the analysis, the points inside the 1 km boundary around ports were removed. Distance from port was determined based on a dataset from 
GFW, which contains anchorages that are either known ports, or that contained at least 20 unique stationary vessels since 2012 \citep{gfw_distance}. 

\section{Data analysis}
To identify persistent hotspot areas throughout the one decade study time and analyse spatio-temporal trends, the Emerging Hotspot Analysis (EHA) tool
was used in ArcGIS Pro \citep{arcgis}. For this, multidimensional netCDF (network common data form) files of the fishing hours data were generated after curation and filtering in R, using
the \textit{terra} package (v1.8.18; \citeauthor{terra_package}, \citeyear{terra_package}). Subsequently, this data was read in as multidimensional raster files in ArcGIS Pro, with the
dimensions corresponding to longitude, latitude and time. Data was aggregated annually, using the sum of fishing hours per year for each cell.
From this, a space-time cube, which is the input required for the EHA tool, was created (Fig.~\ref{fig:cube}A).

\begin{figure}[H]
\begin{subfigure}{0.45\textwidth}
    \tdplotsetmaincoords{60}{45} 
    % the first argument cannot be larger than 90
    \begin{tikzpicture}[tdplot_main_coords, line join=round, 3d cube/.cd,
        num cubes x=4, num cubes y=4, num cubes z=4]
        \node[font=\large] at (-8,1) {(A)};
        \path pic {cube array={}};

        Longitude, Latitude, Time) aligned with cube edges
        % Make sure you're inside this scope:
        \begin{scope}[tdplot_main_coords]
            
            % Arrow: Longitude (X-axis): from left-bottom-front to middle-bottom-front
            \draw[->, thick, orangeBase]
            (-2.5, -2.5, -2.5) -- (3, -3, -2)
            node[midway, below, sloped] {Longitude};
            
            % Arrow: Latitude (Y-axis): from middle-bottom-front to right-bottom-front
            \draw[->, thick, grayLeaf]
            (3, -3, -2) -- (3, 2, -2)
            node[midway, below, sloped] {Latitude};
            % Arrow: Time (Z-axis): from bottom-left to top-left
            \draw[->, thick, greenBase]
            (-2.5, -2.5, -2.5) -- (-2, -3, 2)
            node[midway, left] {\shortstack{Time\\(Years)}};

            \draw[->, thick, red!70]
            (2,-1.2, 2.4)..controls(0,0.5,2.2)..(0.7,0.7,3.5);
            
            \node[text=black] at (1,1,3.9) {Time series bin};

        \end{scope}
        
    \end{tikzpicture}
\end{subfigure}
\hspace{1.5cm}
    \begin{subfigure}{0.45\textwidth}
        \centering
        \begin{tikzpicture}[scale=.6, every node/.style={minimum size=1cm}]
        \node[anchor=north west, font=\large] at (-4.5,8) {(B)};
        % Define grid and insets
        \def\gridsize{5}
        \def\cellsize{1}
        \def\inset{0.1} % amount to inset the fill from each side
        
        % Function to draw filled cell with inset
        \newcommand{\fillcell}[3]{
            \fill[#3, fill opacity = 0.6] (#1+\inset,#2+\inset) rectangle (#1+1-\inset,#2+1-\inset);
            }
            
        % === First Grid (lower, 3D perspective) ===
        \begin{scope}[yshift=-5cm, xshift=1cm, every node/.append style={yslant=0.5,xslant=-1}, yslant=0.5, xslant=-1]
            \draw[step=\cellsize, black] (0,0) grid (\gridsize,\gridsize);
            \draw[black, thick] (0,0) rectangle (\gridsize,\gridsize);

  % Center cell
  \fillcell{2}{2}{darkgreenMW}

  % 8-neighbourhood
  \foreach \x/\y in {1/1, 1/2, 1/3, 2/1, 2/3, 3/1, 3/2, 3/3} {\fillcell{\x}{\y}{lightgreenMW}}
\end{scope}

% === Arrow (behind top grid) ===
\coordinate (center_tminus1) at (1,3.5-6); % center of bottom grid
\coordinate (center_t) at (0,2.6);       % center of top grid
\draw[-latex, thick, black] (center_tminus1) -- (center_t);
\node[below left, font=\bfseries\large] at (center_tminus1) {$t\!-\!1$};

% === Second Grid (upper, semi-transparent overlay) ===
\begin{scope}[yshift=0cm, every node/.append style={yslant=0.5,xslant=-1}, yslant=0.5, xslant=-1]
  \fill[white, fill opacity=0.4] (0,0) rectangle (\gridsize,\gridsize);
  \draw[step=\cellsize, black] (0,0) grid (\gridsize,\gridsize);
  \draw[black, thick] (0,0) rectangle (\gridsize,\gridsize);

  % Center cell
  \fillcell{2}{2}{darkgreenMW}

  % 8-neighbourhood
  \foreach \x/\y in {1/1, 1/2, 1/3, 2/1, 2/3, 3/1, 3/2, 3/3} {\fillcell{\x}{\y}{lightgreenMW}}
\end{scope}

% === Annotations ===
\node[font=\bfseries\large] at (-0.5,2.7) {$t$}; % Position manually above the top grid

        \end{tikzpicture}
    \end{subfigure}
\bigskip
\caption{A) Structure of the space-time cube in ArcGIS\@. Each individual cube corresponds to one \textit{neighbourhood bin}, which is the sum of all coloured cells in B. One \textit{time series bin}
corresponds to the same location over time (red).  B) Conceptualization of space-time dependency as implemented in the Emerging Hotspot Analysis tool. One \textit{neighbourhood  bin}
is defined as the cell itself (darkgreen) plus the cells surrounding it (lightgreen), as well as those cells in the previous time step (\(t-1\)).}
\label{fig:cube}
\end{figure}

EHA uses a combination of two statistical methods. First, the Getis-Ord \(G_i^*\) statistic to identify areas where low/high values are spatially clustered \citep{getis_ord}. The null hypothesis
states that the sum of values of location \( i \) and its neighbours, is not significantly different from what would be expected by chance, based on all observations (neighbours are
defined as shown in Fig.~\ref{fig:cube}B). Thus, each \textit{neighbourhood} contains the cell itself, plus all cells contiguous with it via edges and corners at time \(t\) and \(t-1\).
Each neighbourhood is compared to all global observations at the current and preceding time step. Based on the neighbourhood definition,
a binary spatial weight matrix is constructed, where each entry \(w_{i,j}\), is either 1 (if features \(i\) and \(j\) are neighbours) or 0 otherwise. The \(G_i^*\) statistic is then
calculated as:

\begin{equation}
G_i^* = \frac{\displaystyle\sum_{j=1}^{n} w_{i,j} x_j - \bar{X} \displaystyle\sum_{j=1}^{n} w_{i,j}}{S \sqrt{\frac{n \displaystyle\sum_{j=1}^{n} w_{i,j}^2 - \left( \displaystyle\sum_{j=1}^{n} w_{i,j} \right)^2}{n-1}}}
\end{equation}

\bigskip

where \( x_j \) is the value for feature \( j \), 
\( w_{i,j} \) is the spatial weight between feature \( i \) and \( j \), 
and \( n \) is the total number of features. The terms \( \bar{X} \) and \( S \) represent the global mean and standard deviation
of the attribute values, respectively, and are given by:

\begin{equation}
\bar{X} = \frac{\displaystyle\sum_{j=1}^{n} x_j}{n}
\end{equation}
\medskip
\begin{equation}
S = \sqrt{\frac{\displaystyle\sum_{j=1}^{n} x_j^2}{n} - \left(\bar{X}\right)^2}
\end{equation}\\

The implementation of \(G_i^*\) in the EHA tool also applies a False Discovery Rate (FDR) correction to account for multiple testing and spatial dependency in the data.
This approach is preferred over methods like Bonferroni correction, which only accounts for multiple testing, as FDR is less conservative and less likely to miss true 
positives \citep{fdr_correction}. EHA is thus, a spatio-temporal extension of the \(G_i^*\) statistic, as it extends each cell not only to its spatial but
also to the temporal neighbours.

\bigskip

Second, EHA applies the Mann-Kendall trend test to evaluate whether there is a monotonic upward or downward trend in each time series bin \citep{mann1945nonparametric,kendall1990rank}.
The non-parametric Mann-Kendall statistic \(S\) analyses each time series bin. It ranks and compares each point \( x_i \) (for \( i = 1, 2, \ldots, n-1 \)) 
to all subsequent points \( x_j \) (for \( j = i+1, i+2, \ldots, n \)) and is given by (\citeauthor{kendall1990rank}, \citeyear{kendall1990rank}, Section 1.9)

\begin{equation}
S = \sum_{i=1}^{n-1} \sum_{j=i+1}^{n} \text{sign}(x_j - x_i)
\end{equation}

\bigskip

Where the sign function is defined as:

\begin{equation}
\text{sign}(x_j - x_i) = \text{sign}(R_j - R_i)
\begin{cases}
\phantom{-}1 & x_i < x_j \\
\phantom{-}0 & x_i = x_j \\
  -1 & x_i > x_j
\end{cases}
\end{equation}

\bigskip

and \(R_i\) and \(R_j\) are the ranks of observations \(x_i\) and \(x_j\) of each time series.
Thus, for every time point, it assigns a 1 if the value is higher than the previous one, a 0 if the value is the same, and a -1 if the value is lower. These scores are
then summed for each time series bin and under the null hypothesis of no trend, the value of \(S\) is zero. To assess the significance of \(S\), the variance \(V^*_0\)
can be calculated as (\citeauthor{kendall1990rank}, \citeyear{kendall1990rank}, Section 4.9)

\begin{equation}
    V^*_0 (S) = n(n - 1)(2n + 5) / 18 - \sum_{j=1}^{m} t_j(t_j - 1)(2t_j + 5) /18
\end{equation}

\bigskip

where \(n\) is the total number of observations, and \(m\) the number of groups with tied ranks, each with \(t_j\) tied observations. 
