\chapter{Introduction}

\begin{itemize}
    \item Delineating fishing grounds essential for fisheries management
    \item Traditional tools and shortcomings, logbooks, VMS (resolution, ping rate etc.)
    \item AIS as promising tool to understand fisheries dynamics for example to estimate fishing effort or to aid in fisheries governance (MCS)
    \begin{itemize}
        \item Explain how AIS works
    \end{itemize}
    \item Limitations of AIS
    \item Conservation issues in Mediterranean, highly migratory species
    \begin{itemize}
        \item New approach in ICCAT to go towards climate informed fisheries
        \item TunaMed Observatory
        \item Monitor changes in spatio-temporal distribution essential
        \item Develop indicators for habitat changes due to climate change but also indicators for human pressures like fishing
    \end{itemize}
    \item BFT collapse and exploitation history, tuna farms
    \item Countries report coarse spatial data to ICCAT, if any
    \item AIS allows fine-scale analysis of spatio-temporal changes
    \item Metiers for each species, when they are fished, how they migrate
    \item Research questions:
    \begin{itemize}
        \item Where and when is fishing activity for large pelagic species most intense and has this changed over time from 2015-2024?
        \item What are the seasonal patterns of fishing activity by gear type and do they differ interannually?
        \item To what extent does AIS capture the full scope of fishing activity in the Mediterranean?
        \item What is the spatial relationship between fishing hours and environmental features such as distance to port, and bathymetry
              and how does this differ between the two gear types?
        \item How: Fishing hours data from GFW etc.

    \end{itemize}
\end{itemize}

